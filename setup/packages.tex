% Declare any packages and options you are using here.

% ACTIVE PACKAGES %%%%%%%%%%%%%%%%%%%%%%%%%%%%%%%%%%%%%%%%%%%%%%%%%%%%%%%%%%%%%%
% Used packages. Keep packages in alphabetical order, except for hyperref below.
%
% Use acronyms (define actual acronyms in setup/acro.tex)
\usepackage{acro}
% Make tables look nicer
\usepackage{booktabs}
% Compress and sort multi-citations
\usepackage[sort]{cite}
% Better enumeration, including inline lists
\usepackage[inline]{enumitem}
% Use 1-inch margins (PAPPG 23-1, I.II.C.2.c)
\usepackage[margin=1in]{geometry}
% Allow figures
\usepackage{graphicx}
% Use Times New Roman font (PAPPG 23-1, I.II.C.2.a)
\usepackage{mathptmx}
% Enable table columns to fill/wrap a specific width
\usepackage{tabularx}
% Enable formatting of paper/section titles
\usepackage{titlesec}
% Define colors for comments, etc.
\usepackage{xcolor}
% For spacing after user-defined commands and others
\usepackage{xspace}
% This has to be loaded last because it modifies a lot of other commands.
\usepackage{hyperref}  % Allow hyperlinked references
%%%%%%%%%%%%%%%%%%%%%%%%%%%%%%%%%%%%%%%%%%%%%%%%%%%%%%%%%%%%%%%%%%%%%%%%%%%%%%%%

% OTHER PACKAGES %%%%%%%%%%%%%%%%%%%%%%%%%%%%%%%%%%%%%%%%%%%%%%%%%%%%%%%%%%%%%%%
% Packages that are not used, but may be useful in certain situations. Move and
% uncomment to the list above to use.
%
% Enable algorithm descriptions in pseudocode.
%\usepackage{algorithmic}
%
% Enable features useful for typesetting mathematical notation (e.g., multiline
% alignment for equations).
%\usepackage{amsmath}
%
% Load AMS fonts.
%\usepackage{amsfonts}
%
% Enable commands for certain mathematical symbols (e.g., \varnothing for the
% empty set).
%\usepackage{amssymb}
%
% Get bitmap size and resolutions
%\usepackage{bmpsize}
%
% Enable message sequence charts, which are useful for showing messages sent in
% a particular order (e.g., in network protocols).
%\usepackage{msc}
%
% Enable multiple subfigures within a single figure.
%\usepackage{subfig}
%
% Enable various text symbols (e.g., currency symbols, bullet points, copyright
% symbols, musical notation).
%\usepackage{textcomp}
%
% Make URLs display correctly in citations. The hyphens option allows line
% breaks after hyphens.
%\usepackage[hyphens]{url}
%%%%%%%%%%%%%%%%%%%%%%%%%%%%%%%%%%%%%%%%%%%%%%%%%%%%%%%%%%%%%%%%%%%%%%%%%%%%%%%%
